%   Copyright (C) 2014 Reinventing Geospatial, Inc
%
%   This program is free software: you can redistribute it and/or modify
%   it under the terms of the GNU General Public License as published by
%   the Free Software Foundation, either version 3 of the License, or
%   (at your option) any later version.
%
%   This program is distributed in the hope that it will be useful,
%   but WITHOUT ANY WARRANTY; without even the implied warranty of
%   MERCHANTABILITY or FITNESS FOR A PARTICULAR PURPOSE.  See the
%   GNU General Public License for more details.
%
%   You should have received a copy of the GNU General Public License
%   along with this program.  If not, see <http://www.gnu.org/licenses/>,
%   or write to the Free Software Foundation, Inc., 59 Temple Place -
%   Suite 330, Boston, MA 02111-1307, USA.

\documentclass{article}
\usepackage{multirow}
\usepackage[margin=1in]{geometry}
\usepackage[hidelinks]{hyperref}
\usepackage{listings}
\usepackage{courier}
\usepackage{fancyhdr}

\lstset{basicstyle=\ttfamily}

\pagestyle{fancy}
\rfoot{Version Number: 1.0}

\title{Instructions For gdal2tiles\_parallel.py}
\author{Steven D. Lander, Reinventing Geospatial}

\begin{document}
\maketitle

\section{Purpose}
This document covers installation of dependencies, in-depth explanations of
command-line arguments, and usage examples for gdal2tiles\_parallel.py.

The file gdal2tiles\_parallel.py is a Python script that converts
GDAL-supported raster imagery files into a folder of tiles in TMS format
(z/x/y).  This version improves upon the standard gdal2tiles.py file included
with GDAL by adding multiprocessing improvements to obtain even better tile
generation performance.

\section{Installing Dependencies}
Gdal2tiles\_parallel.py was written for Python 2.7.x and has not been tested on
any other version.  It will run on either 32 or 64 bit systems with no issues.

The script relies on Python 2.7.x, GDAL core, and the GDAL Python bindings.
Some imagery formats such as MrSID will require additional driver installers to
work correctly.
\subsection{Windows}
In order to run gdal2tiles\_parallel.py on a Windows environment, install
the following packages.  For convenience, they have been included in all
PythonGeopackage releases and are viewable in source under Dependencies.
\begin{itemize}
    \item
        The latest version of Python 2.7 for either x86 (32 bit) or x86\_64
        (64 bit): 
        \url{https://www.python.org/downloads/windows/}.
    \item
        The lastest stable release of the Geospatial Data Abstraction Library
        (GDAL) in either x86 (32 bit) or x86\_64 (64 bit):
        \url{http://www.gisinternals.com/sdk}
\end{itemize}
\subsection{Linux}
In order to run gdal2tiles\_parallel.py on a Linux environment, instructions
will differ slightly by distribution.  Most newer Linux distributions have
Python 2.7 as either an option or the default python version in their package
repository, but others such as CentOS only have older versions.  In that case,
the user will need to find out how to get Python 2.7 from a reputable source or
compile it themselves.  The following packages are needed:
\begin{itemize}
    \item 
        The latest version of Python 2.7 for your system.  For Debian-based
        distributions such as Ubuntu and RedHat type the following into a
        command line: 

        \lstinline|sudo apt-get install python2.7 python2.7-dev  python-pip base-devel| 

        This will install the Python 2.7 environment plus PIP, a Python
        module manager.
    \item
        The latest GDAL binaries and python bindings for your system.  For
        Debian-based distributions such as Ubuntu and RedHat, type the
        following into a command line:

        \lstinline|sudo apt-get install gdal-bin python-gdal|
\end{itemize}
\section{Usage}
\subsection{Command Line Arguments}
Gdal2tiles\_parallel.py supports additional functionality via command line
arguments provided to the script at the time it is executed.  Following is a
outline of the important flags:
\\\\
\begin{tabular}{ | c | p{14cm} | }
    \hline
    -h, --help & Print the listing of commands available for the script.\\
    \hline
    -p, --profile & Specify the tiling profile you would like these tiles to be
    created in.  Valid options are mercator or geodetic.\\
    \hline
    -e, --resume & Instruct the script to not overwrite tiles that have already
    been created.  This is a {\bf mandatory} flag when using default
    multiprocessing.\\
    \hline
    -z, --zoom & The zoom levels to create.  Allows the tiler to make tiles
    past the default zoom level that GDAL detects.  (Format: '2-5' or '10')
    \\
    \hline
    -a, --srcnodata & Specify the RGB value that gdal2tiles should convert to
    transparency.  Typical value should be '0,0,0'.
    \\
    \hline
\end{tabular}

\subsection{Examples}
\begin{itemize}
    \item
        Create a folder of tiles in the mercator projection based on a GeoTiff
        image named WhiteHorse.tif and name the folder 'WhiteHorse\_tiles':\\
        \\
        \lstinline|python gdal2tiles_parallel.py|\\
        \lstinline|    -p mercator -e|\\
        \lstinline|    /data/raw/WhiteHorse.tif /data/tiles/mercator/WhiteHorse_tiles|\\
    \item
        Create a folder of tiles for zoom level 15 in the mercator projection
        based on a GeoTiff image named WhiteHorse.tif and name the folder
        'WhiteHorse\_tiles':\\
        \\
        \lstinline|python gdal2tiles_parallel.py|\\
        \lstinline|    -p mercator -e -z 15|\\
        \lstinline|    /data/raw/WhiteHorse.tif /data/tiles/mercator/WhiteHorse_tiles|\\
    \item
        Create a folder of tiles in the geodetic projection based on a MrSID
        image named FortBelvoir\_201307\_A6.sid and name the folder
        belvoir\_tiles. Also, assigns the NODATA transparency to the RGB color
        value of 0,0,0:\\
        \\
        \lstinline|python gdal2tiles_parallel.py|\\
        \lstinline|    -p geodetic -e -a 0,0,0|\\
        \lstinline|    /data/raw/FortBelvoir_201307_A6.sid /data/tiles/belvoir_tiles|
    \item
        Create a folder of tiles in the geodetic projection based on a MrSID
        image named FortBelvoir\_201307\_A6.sid and name the folder
        belvoir\_tiles. Also, assigns the NODATA transparency to the RGB color
        value of 255,0,0:\\
        \\
        \lstinline|python gdal2tiles_parallel.py|\\
        \lstinline|    -p geodetic -e -a 255,0,0|\\
        \lstinline|    /data/raw/FortBelvoir_201307_A6.sid /data/tiles/belvoir_tiles|
    \item
        Create a folder of tiles for zoom levels 10 through 13 in the geodetic
        projection based on a MrSID image named FortBelvoir\_201307\_A6.sid
        and name the folder belvoir\_tiles. Also, assigns the NODATA
        transparency to the RGB color value of 0,0,0:\\
        \\
        \lstinline|python gdal2tiles_parallel.py|\\
        \lstinline|    -p geodetic -e -a 0,0,0|\\
        \lstinline|    -z 10-13|\\
        \lstinline|    /data/raw/FortBelvoir_201307_A6.sid /data/tiles/belvoir_tiles|
\end{itemize}

\section{Caveats \& Known Issues}
\begin{itemize}
    \item
        Gdal2tiles\_parallel.py currently only outputs tiles as full-color
        PNGs.  This is so that transparency can be preserved.  When used in
        conjunction with tiles2gpkg\_parallel.py a user can make a geopackage
        with either PNGs, JPEGs, or a mixture of both.
    \item
        Since this script utilizes multiprocessing, it does not exit cleanly
        when interrupted.  For example, if Ctrl+D (KeyboardInterrupt) while the
        script is executing, it will not completely stop but instead produce
        profuse error messages.  To kill the process, close the terminal
        windows on Windows or, on Linux, send the process to the background
        (Ctrl+Z) then kill the job with \lstinline|kill -9 $(jobs -p)|.
    \item
        Gdal2tiles\_parallel.py only creates tiles with a lower-left tile
        origin.
\end{itemize}

\end{document}
